\documentclass[40pt]{article}
\usepackage{babel}
\usepackage[T1]{fontenc}
\usepackage{textcomp}
\usepackage[utf8]{inputenc} % Puede depender del sistema o editor
\usepackage{enumerate}


\title{\textbf{Universidad Veracruzana} }
\date{\textbf{Facultad de Negocios y Tecnologias} }


\begin{document}
\maketitle
%\section{Integrantes}
\textsf{\Large Experiencia Educativa: Administracion de Software. \\}
 
\maketitle
%\section{Integrantes}
\textsf{\Large Catedratico: Centeno Tellez Adolfo. \\}

\maketitle
%\section{Integrantes}
\textsf{\Large Tema: Plan de Comunicacion. \\}

\maketitle
%\section{Integrantes}
\textsf{\Large Integrantes: \\}
\begin{itemize}
    \item Basilio Hernandez Jahaziel.
    \item Hernandez Sanchez Jesus Gabriel.
    \item Jimenez Milan Jose Alfredo.
    \item Perez Castro David.
    \item Zarate Espinosa Jose Pedro.   
\end{itemize}

\maketitle
%\section{Integrantes}
\textsf{\ Grupo: 503 ISW   1° Parcial \\}

\maketitle
%\section{Integrantes}
\textsf{\ Fecha de Entrega: 13 de Octubre del 2020 \\}

\newpage

\maketitle
%\section{Integrantes}
\textsf{\ \\
\textbf{Proposito:}\\
\\
En todos los equipos de trabajo suele haber problemas de comunicación e interpretación, por lo cual como futuros ingenieros nuestro deber no es solo resolverlos si no prevenirlos, ¿cómo podemos hacer esto? muy fácil realizando planes para un proceso contemplando casos de riesgo, en este caso para nuestro proyecto realizaremos un plan de comunicación. \\}

\maketitle
%\section{Integrantes}
\textsf{\ \\
\textbf{Objetivo:}\\
\\
En este proyecto que estamos realizando pondremos a prueba los conocimientos que obtuvimos de la E.E (Experiencia Educativa) Administración de Proyectos de Software que a su vez va de la mano de la E.E Pruebas de Software, como objetivo tenemos que implementar un plan de comunicación que no solo sea eficaz si no que sea productivo, por lo cual demostraremos el manejo de distintas plataformas ofimáticas de comunicación para poder logar un proyecto de buena calidad. \\}

\maketitle
%\section{Integrantes}
\textsf{\ \\
\textbf{Objetivos a corto plazo:}\\
\\
Es indispensable plantear los objetivos a corto plazo ya que sin estos nuestro proyecto pudiera tener bases deficientes, a continuación, se enlistan: \\}
\begin{itemize}
    \item Definir el rol de cada uno de los integrantes del equipo (Lider, Equipo de trabajo).
    \item Realizar un plan de comunicacion eficiente.
    \item Plantear los medios de comunicacion para el equipo.
    \item Lograr una comunicacion efectiva mediante reuniones establecidas en el plan de comunicacion.
    \item Definir el tema del proyecto.
    \item Plantear los objetivos de la primera version.
    \item Poner en marcha la version inicial del proyecto.
\end{itemize}

\maketitle
%\section{Integrantes}
\textsf{\ \\
\textbf{Objetivos a largo plazo:}\\
\\
Los siguientes objetivos que se mostraran se pueden llevar a cabo una vez realizados los anteriores ya mencionados, ya que estos se podrían definir como una continuación de los objetivos a corto plazo: \\}
\\

\begin{itemize}
    \item Tener buena comunicacion durante todo el proyecto.
    \item Realizar mejoras al plan de comunicacion segun lo requiera o necesite.
    \item Verificar el cumplimiento con los objetivos del proyecto.
    \item Terminar en tiempo y forma el proyecto.
    \item Brindar soporte a la version final cuando lo requiera.
\end{itemize}

\maketitle
%\section{Integrantes}
\textsf{\ \\
\textbf{Plan de Comunicacion:}\\
\\
A continuacion se presentan los medios de comunicacion que ocuparemos para trabajar: \\}
\\
\\
\begin{tabular}{| c | c | c |} \hline
\textbf{Descripcion} & \textbf{Medio de Comunicacion} & \textbf{Uso} \\ \hline
Revision del estado del proyecto & Google Meet & Diariamente \\ \hline
Dudas y Aclaraciones & Whatsapp y Discord & Diariamente \\ \hline
Actualizacion de repositorio & Github y Servidor & Cuando se requiera \\ \hline
\end{tabular}
\end{document}
