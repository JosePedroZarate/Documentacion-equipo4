\documentclass[40pt]{article}
\usepackage{babel}
\usepackage[T1]{fontenc}
\usepackage{textcomp}
\usepackage[utf8]{inputenc} % Puede depender del sistema o editor
\usepackage{enumerate}


\title{\textbf{Universidad Veracruzana} }
\date{\textbf{Facultad de Negocios y Tecnologias} }


\begin{document}
\maketitle
%\section{Integrantes}
\textsf{\Large Experiencia Educativa: Administracion de Software. \\}
 
\maketitle
%\section{Integrantes}
\textsf{\Large Catedratico: Centeno Tellez Adolfo. \\}

\maketitle
%\section{Integrantes}
\textsf{\Large Tema: Vision y Alcance de un Proyecto de Software. \\}

\maketitle
%\section{Integrantes}
\textsf{\Large Integrantes: \\}
\begin{itemize}
    \item Basilio Hernandez Jahaziel.
    \item Hernandez Sanchez Jesus Gabriel.
    \item Jimenez Milan Jose Alfredo.
    \item Perez Castro David.
    \item Zarate Espinosa Jose Pedro.   
\end{itemize}

\maketitle
%\section{Integrantes}
\textsf{\ Grupo: 503 ISW 1° Parcial \\}

\maketitle
%\section{Integrantes}
\textsf{\ Fecha de Entrega: 13 de Octubre del 2020 \\}

\newpage

\maketitle
%\section{Integrantes}
\textsf{\ \\
\textbf{Introducción:}\\
\\
Este documento se elabora para la visión y alcance de este proyecto de Accidentes en EE. UU. En el cual se requiere poner a prueba los conocimientos que hemos obtenido en las E.E (Experiencias Educativas) de Administración de Software y Pruebas de Software.\\
\\
Este es un conjunto de datos de accidentes automovilísticos de todo el país, que cubre 49 estados de EE. UU. Los datos de accidentes se recopilan desde febrero de 2016 hasta junio de 2020, utilizando dos API que proporcionan datos de incidentes (o eventos) de tráfico en tiempo real. Estas API transmiten datos de tráfico capturados por una variedad de entidades, como los departamentos de transporte de los Estados Unidos y el estado, agencias de aplicación de la ley, cámaras de tráfico y sensores de tráfico dentro de las redes de carreteras. \\}

\maketitle
%\section{Integrantes}
\textsf{\ \\
\textbf{Recursos / Definiciones:}\\
\\
\textbf{HTML:} Es un lenguaje de marcado que se utiliza para el desarrollo de páginas de Internet. Se trata de las siglas que corresponden a HyperText Markup Language, es decir, Lenguaje de Marcas de Hipertexto.
\\
\\
\textbf{TypeScrip:} Es un lenguaje de programación libre y de código abierto desarrollado y mantenido por Microsoft. Es un superconjunto de JavaScript, que esencialmente añade tipos estáticos y objetos basados en clases.
\\
\\
\textbf{JavaScrip:} Es un lenguaje de programación interpretado, dialecto del estándar ECMAScript. Se define como orientado a objetos,  basado en prototipos, imperativo, débilmente tipado y dinámico.
\\
\\
\textbf{SCSS:} Es un lenguaje de hoja de estilos inicialmente diseñado por Hampton Catlin y desarrollado por Natalie Weizenbaum.Después de sus versiones iniciales, Nathan Weizenbaum y Chris Eppstein han continuado extendiendo Sass con SassScript, un lenguaje de script simple, usado en los ficheros Sass. \\}

\maketitle
%\section{Integrantes}
\textsf{\ \\
\textbf{Planificación del Proyecto:}\\
\\
En este apartado es importante definir las actividades que se realizaran a lo largo del proyecto, lo cual ira de una breve descripción de lo que tratara, asi como su fecha de elaboración, o límite de vencimiento. \\}

\vspace{0.5 cm}
\begin{table}[h!]
\begin{tabular}{|p{4.5 cm}|p{4 cm}|p{4 cm}|}
\hline
\textbf{Actividad:}&\textbf{Descripcion:}&\textbf{Fecha(s):}
\\\hline
Planificación & Plantear por etapas o me- tas los objetivos del pro- yecto. & 02/10/2020 \\\hline
Definir el Equipo de trabajo & Reunir un grupo de per- sonas para formar el equi- po de trbajo max 6 inte- grantes. & 03/10/2020 \\\hline
Documentación & Debe estar organizada, te- ner veracidad y ser clara para que cualquier usuario pueda entender.  & 03/10/2020 - 04/10/2020. \\\hline
Desarrollo & Etapa terminada, para poder realizar pruebas y corregir pequeños deta- lles antes de entrega final de la versión. & 03/10/2020 - 07/10/2020. \\\hline
Pruebas & Cero errores, realización de pruebas de verificación de los requerimientos de la versión realizada. & 05/10/2020 - 08/10/2020. \\\hline
Entregas & Entrega del sistema ter- minado y en su total fun- cionabilidad. & 13/10/2020.\\\hline
\end{tabular}
\end{table}

\maketitle
%\section{Integrantes}
\textsf{\ \\
\textbf{Perfil del Equipo:}\\
\\
Los integrantes del equipo tienen participación activa durante el tiempo que dure el proyecto, cada acción a realizar tendrá un impacto en el proyecto ya sea negativa o positiva, por lo cual les mostramos los perfiles de los individuos que conforman este equipo de trabajo en la siguiente tabla: \\}

\begin{table}[h!]
\begin{tabular}{|p{4.5 cm}|p{4 cm}|p{4 cm}|}
\hline
\textbf{Integrante(s):}&\textbf{Tarea(s) a cumplir:}&\textbf{Restricciones:}
\\\hline

Basilio Hernandez Jahaziel & Líder: Gestionar. Admi- nistrar y Supervisar el de- sarrollo del proyecto. & Supervisa y Verifica el e- quipo, implementa el mo- delo branching. \\\hline

Hernandez Sanchez Jesus Ga- briel & Realiza el diseño contem- plado de los mockups, que representara las funciones del sistema. & Utilizara software especia- lizado para la elabora- ción de mockups. \\\hline

Jimenez Milan Jose Alfredo & Busca y analiza Bases de Datos para el proyecto. & De 1 millón de registros en adelante compatible y disponible para su uti- lización. \\\hline

Perez Castro David & Creación del servidor de producción y publicación en ambos servidores. & Servidor establecido para testing por el profesor y el de producción podría ser G Cloud y Azure. \\\hline

Zarate Espinosa Jose Pedro & Elaboración de documen- tación y Diagramas de arquitectura especializa- dos para desarrollar y de- finir el sistema. & Documentos enfocados en las platillas otorgadas por el profesor y softwa- re especializado para la elaboración de arquitectu- ra del sistema. \\\hline
\end{tabular}
\end{table}

\newpage

\maketitle
%\section{Integrantes}
\textsf{\ \\
\textbf{Revisiones:}\\
\\
A continuación, se muestra una tabla con el historial de modificaciones que ha sufrido este documento. \\}

\vspace{0.5 cm}
\begin{table}[h!]
\centering
\begin{tabular}{|p{0.35\linewidth}|p{0.15\linewidth}|p{0.35\linewidth}|p{0.15\linewidth}|}
\hline
\textbf{Nombre}&\textbf{Fecha}&\textbf{Razón del cambio}&\textbf{Versión}
\\\hline
Zarate Espinosa Jose Pedro & 03/10/2020 & Revisión inicial & v0.1 \\\hline
Zarate Espinosa Jose Pedro & 03/10/2020 & Planteamiento de la estruc- tura inicial del documento & v0.2 \\\hline
Zarate Espinosa Jose Pedro & 03/10/2020 & Agregacion de los requeri- mientos del sistema & v0.3 \\\hline
Zarate Espinosa Jose Pedro & 6/10/2020 & Agregacion de tabla de revi- siones & v1.0 \\\hline
Zarate Espinosa Jose Pedro & 10/10/2020 & Modificacion de fecha de en- trega & v1.1 \\\hline
\end{tabular}
\end{table}

\newpage

\maketitle
%\section{Integrantes}
\textsf{\ \\
\textbf{Definición del Negocio (Proyecto):} \\
\\
\textbf{Escenario:}\\
\\
El proyecto Accidentes en EE. UU. Esta contemplado para mostrar accidentes del país ya mencionado, ¿Cómo se logrará?, ¿Cómo estará conformado?, estas son algunas dudas muy comunes, por lo cual el sistema consiste en mostrar un login en el cual se podrá iniciar session mediante su usuario y contraseña,  en caso de no estar dado de alta este mismo te permitirá registrarte para poder ingresar a la segunda conformación la cual muestra un mapa donde el usuario final podrá visualizar marcas en forma de circulo y al dar clic en estas indicara un punto de accidente registrado, el cual tendrá los siguientes datos: id, State, City, Star Time, End time and Description, por lo cual se espera una experiencia mejorada. \\}

\maketitle
%\section{Integrantes}
\textsf{\ \\
\textbf{Objetivo:} \\
\\
Como cualquier negocio rentable, este tendrá el objetivo de ser llamativo y expresivo para una utilización fácil de manejar, por lo cual estará publicado en un servidor (G Cloud), el cual todo usuario podrá acceder ya este dado de alta o registrándose y después dirigiéndolo a una pagina donde se encuentra el mapa de accidentes el cual será muy intuitivo de utilizar e interpretar. \\}

\maketitle
%\section{Integrantes}
\textsf{\ \\
\textbf{Negociaciones del Sistema:} \\
\\
Como se esperaba, este proyecto puede ser rentable para cualquier cliente que muestre interés ya que podría mostrar información acerca de los accidentes de cualquier país mediante la obtención de su base de datos, lo cual generaría nuevas experiencias de búsqueda de siniestros de los estados, cabe recalcar que este sistema cuenta con seguridad de acceso mediante un login, lo cual lo hace aún más valioso y rentable.\\
\\
\textbf{Aclaración:} no se busca lucrar con este sistema, la parte negociable de este documento es solo conformación de la visión y alcance del proyecto, el cual es de ámbito educativo, esta aclaración surge por la base de datos utilizada en este sistema ya que especifica que acepta utilizar estos datos solo para aplicaciones académicas, de investigación o no comerciales. 
 \\}

\maketitle
%\section{Integrantes}
\textsf{\ \\
\textbf{Impacto en el Mercado:} \\
\\
Cada vez son más las necesidades que se deben cubrir, en este sistema se estima que satisfaga la mayoría de necesidades de los usuarios finales por lo cual se implementa la atracción e interacción visual para obtener la información solicitada.\\
\\
Dentro del mercado se espera entrar en un top de 20 mejores sistemas utilizados en los primeros días del lanzamiento y mantenerse dentro de ese top. 
 \\}
 
\maketitle
%\section{Integrantes}
\textsf{\ \\
\textbf{Cumplimiento de Requisitos:} \\
\\
Este sistema deberá cumplir con todos sus objetivos planeados anteriormente para que tenga una funcionabilidad favorable, a continuación, se enlistan los requisitos que debe cumplir el sistema: \\} 
\begin{itemize}
    \item Acceder desde cualquier navegador al sistema.
    \item Inicio de sesión de usuarios dados de alta previamente. 
    \item Registro de usuarios nuevos.
    \item Enlazar a la pagina donde se mostrará la información de accidentes.
    \item Mostrar un mapa con marcas (círculos) el cual a dar clic proporcione la información.   
\end{itemize}

\maketitle
%\section{Integrantes}
\textsf{\ \\
\textbf{Requerimientos Funcionales y No Funcionales:} \\
\\
Un requisito funcional define una función del sistema de software o sus componentes. Una función es descrita como un conjunto de entradas, comportamientos y salidas.\\
\\
Un requisito no funcional o atributo de calidad es, un requisito que sabe bien y especifica criterios que pueden usarse para juzgar la operación de un sistema, a continuación, se presentan: \\}

\textbf{Requerimientos Funcionales:}
\begin{itemize}
    \item Se podrá acceder al sistema desde algún navegador.
    \item Ingresar al sistema a través de un proceso de autentificación. 
    \item El usuario podrá registrarse.
    \item El usuario no tendrá acceso al sistema sin estar registrado.
    \item Utilizar la base de datos para mostrar la información en un mapa.
    \item Navegación por el mapa para consultar la información.
\end{itemize}

\newpage

\textbf{Requerimientos No Funcionales:}
\begin{itemize}
    \item Acceso al sistema, deberá ser de forma segura.
    \item Interfaz intuitiva y completa para el fácil manejo de los usuarios. 
    \item Tiempo para mostrar la información en el mapa no mayor a 45 segundos.
    \item Administración del proyecto en git.
    \item Capacidad de conectarse a un api.
    \item Implementar Kanban automatizados para la administración del repositorio.
    \item Autorización del líder para cada pull request.
\end{itemize}

\maketitle
%\section{Integrantes}
\textsf{\ \\
\textbf{Alcance del Proyecto:} \\
\\
El alcance de esta versión es inicial es con el objetivo de calificar el primer parcial en la E.E Administración de Software, donde definimos las bases que tendrá nuestro sistema como lo es la arquitectura en la que trabajara en futuras versiones, asi mismo poner a prueba los conocimientos que obtuvimos. \\}

\maketitle
%\section{Integrantes}
\textsf{\ \\
\textbf{Los siguientes puntos a mostrar son los alcances de esta primera versión:} \\}

\begin{itemize}
    \item Diseño completo del sistema.
    \item Funcionabilidad de los requerimientos establecidos.
    \item Issues definidos para esta primera version.
    \item Implementación de Hard-code para el sistema.
    \item Realización de Prueba de cada componente y modulo.
    \item Servidores de Testing y Producción levantados y funcionando correctamente.
    \item Entrega de la primera versión el día 13/10/2020.
\end{itemize}

\maketitle
%\section{Integrantes}
\textsf{\ \\
\textbf{Los siguientes puntos a mostrar es el posible alcance de las versiones posteriores:} \\}

\begin{itemize}
    \item Utilización de Apis para implementar el login des sistema.
    \item Seguimiento de la base de datos utilizada en la primera versión. 
\end{itemize}

\maketitle
%\section{Integrantes}
\textsf{\ \\
\textbf{Conclusión:}
\\
\\
En esta ocasión nos presentamos con muchas más dificultades que lo habitual, pero las superamos como un equipo de trabajo, el trabajo de cada participante es mas que notorio, incluso hemos aprendido uno del otro, asi fue como se pudo avanzar mediante nuevas tecnologías que se implementaron en este proyecto, en cada apartado se puede notar la colaboración de mas de una persona, ya que todo estuvo sujeto a opiniones y modificaciones todo esto para poder lograr un proyecto de excelencia. \\}



\end{document}